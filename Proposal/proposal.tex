\documentclass{article}
\usepackage{amsfonts, amsmath, amsthm, amssymb}
\usepackage[urlcolor=blue,linkcolor=black,colorlinks=true]{hyperref}
\hypersetup{
  pdftitle = {A Novel Approach to Detecting Covert DNS Tunnels Using Throughput Estimation},
  pdfkeywords = {},
  pdfauthor = {Michael Himbeault}
}
\usepackage[headsep=1cm,headheight=50pt]{geometry}
\usepackage[dvips]{graphicx}
\usepackage{subfigure}
\usepackage{lscape}

\newcommand{\hreff}[2]{\href{#1}{#2}\footnote{\url{#1}}}

\newtheorem{thm}{Theorem}[section]
\newtheorem{cor}[thm]{Corollary}
\newtheorem{lem}[thm]{Lemma}

\newtheorem*{hyp}{Hypothesis}

\theoremstyle{remark}
\newtheorem{remark}[thm]{Remark}

\theoremstyle{definition}
\newtheorem{definition}[thm]{Definition}

\theoremstyle{definition}
\newtheorem{example}{Example}[section]

\theoremstyle{definition}
\newtheorem{algorithm}{Algorithm}[section]

\begin{document}
\bibliographystyle{amsplain}
\title{A Novel Approach to Detecting Covert DNS Tunnels Using Throughput Estimation}
\author{Michael Himbeault}

\maketitle

\abstract{In a world that runs on data, protection of that data and protection of the \emph{motion} of that data is of the utmost importance. Covert communication channels attempt to circumvent established methods of control, such as firewalls and proxies, by utilizing non-standard means of getting messages between two endpoints. DNS (Domain Name System), the system that translates text-based resource names into IP addresses, is a very common and effective platform upon which covert channels are built. This work proposes, and demonstrates the effectiveness of, a novel new technique that estimates data transmission throughput over DNS and its weight against the norm for a given network link to identify the existence of a DNS tunnel. The proposed technique is robust in the face of the obfuscation techniques that are able to hide tunnels from existing detection methods.}

\tableofcontents

\newpage

\section{Problem Statement}

As networks increase in capacity and more applications are producing traffic across the same network connection, the heterogeneity of the traffic on these links is increasing. With this increasing diversity in network traffic comes an ever widening range of acceptable behaviour patterns which results in a very hostile and complicated environment for anomaly detection algorithms.

The goal of this research is to investigate the feasibility of detecting DNS tunnels in near real time at high throughput on a complex network link. This detection method should also be robust against known methods of avoiding existing techniques.

\section{Introduction}

As computing technology has evolved, the amount of information that needs to be protected has grown with it. In 2012, a small ISP (Internet Service Provider) that serves less than ten thousand clients can move several terabytes of data per day\footnote{YouTube video at a resolution of 720p is encoded at a bitrate of approximately 2.5Mbps which translates into approximately fifty megabytes for a three minute video. If a average person watches four such videos in a day, then ten thousand clients can produce two terabytes of traffic in a day.}.

Included in this is data generated by hundreds, if not thousands, of unique applications including web browsers, operating systems, video games, communication programs, office programs and machine-to-machine communications. It is important to realize that each web application (GMail, Yahoo mail, Google Docs, YouTube, etc...) behaves in this respect as a unique application. Each type of traffic from each application has its own specific pattern that describes the traffic it produces under normal circumstances and deviations from this pattern contain important information.

The problem of identifying what portions of the data are either malicious or otherwise abnormal becomes one of both scale and discrimination. When looking at network links sufficiently busy to supply enough information for analysis, there is also a great deal of unrelated traffic that needs to be filtered out in order to ensure an accurate result. Being discriminatory enough detect relatively rare events\footnote{A DNS tunnel, for example, can be as few as ten to twenty packets within billions over a day.} without incurring a large number of false positive is essential for an effective detection method.

Covert communication channels can fall into the category of a 'needle in a hay stack', or they can be very obvious but this depends on, primarily, how much data they are moving. The more data that is being moved across a covert channel, the more likely it is to show up as an anomaly due to the number of packets and the proportion of the total traffic it represents.



% In order to be able to detect these anomalies at a large scale, and to reduce the number of devices performing the analysis on the network, the point at which the interception is done needs to be moved closer to the centre of the network. Throughput on a given link increase rapidly as the chosen link gets closer to the centre of the network, due to aggregation of traffic. 
% 
% Sifting through gigabits of traffic becomes a very computationally expensive problem; one which is normally solved by applying more, or specialized, hardware\footnote{See proprietary solutions, such as those offered by TippingPoint, which employ specialized hardware components such as ASICs and FPGAs to improve performance. Other vendors, such as Wedge Networks products, improve performance by exploiting the ability to perform portions of its workload in parallel and add more commodity processors.}. In order to combat this inevitable march of requirements, 

\end{document}