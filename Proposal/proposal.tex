\documentclass[12pt]{article}
\usepackage{amsfonts, amsmath, amsthm, amssymb}
\usepackage[urlcolor=blue,linkcolor=black,colorlinks=true]{hyperref}
\hypersetup{
  pdftitle = {A Novel Approach to Detecting Covert DNS Tunnels Using Throughput
Estimation},
  pdfkeywords = {},
  pdfauthor = {Michael Himbeault}
}
\usepackage[headsep=1cm,headheight=50pt]{geometry}
\usepackage[dvips]{graphicx}
\usepackage{subfigure}
\usepackage{lscape}

\usepackage{setspace}
\doublespacing

\addtolength{\oddsidemargin}{-.5in}
\addtolength{\evensidemargin}{-.5in}
\addtolength{\textwidth}{1in}

\addtolength{\topmargin}{-.5in}
\addtolength{\textheight}{1in}

\newcommand{\hreff}[2]{\href{#1}{#2}\footnote{\url{#1}}}

\newtheorem{thm}{Theorem}[section]
\newtheorem{cor}[thm]{Corollary}
\newtheorem{lem}[thm]{Lemma}

\newtheorem*{hyp}{Hypothesis}

\theoremstyle{remark}
\newtheorem{remark}[thm]{Remark}

\theoremstyle{definition}
\newtheorem{definition}[thm]{Definition}

\theoremstyle{definition}
\newtheorem{example}{Example}[section]

\theoremstyle{definition}
\newtheorem{algorithm}{Algorithm}[section]

\begin{document}
\bibliographystyle{amsplain}
\title{A Novel Approach to Detecting Covert DNS Tunnels Using Throughput
Estimation}
\author{Michael Himbeault}

\maketitle

\abstract{In a world that runs on data, protection of that data and protection 
of the \emph{motion} of that data is of the utmost importance. Covert
communication channels attempt to circumvent established methods of control, such
as firewalls and proxies, by utilizing non-standard means of getting messages
between two endpoints. DNS (Domain Name System), the system that  translates
text-based resource names into resource records, is a very common and effective
platform upon which covert channels are often built. This work proposes, and
demonstrates the effectiveness of, a novel new technique that estimates data  
transmission throughput over DNS in order to identify the existence of a DNS
tunnel. The proposed technique is robust in the face of the obfuscation techniques
that are able to hide tunnels from existing detection methods.}

\newpage
\tableofcontents

\newpage

% http://www.sce.carleton.ca/faculty/chinneck/thesis.html

\section{Introduction}

Because it is important to control the data that is sent between hosts, systems
such as firewalls, proxies, and content filters are put into place in order to
monitor and control the network traffic. Covert channels, such as DNS tunnels,
are utilized to circumvent these control mechanisms for purposes that range from
benign to malicious. This work focuses specifically on DNS tunnels, however
other types of covert channels do exist.

Detecting a DNS tunnel effectively on a busy network link becomes an exercise
in discrimination. Because there is such a wide variety of network traffic that
is generated on a busy link, there is generally no simple definition of
\emph{normal} for a particular class of traffic, including DNS. This tends to
either rule out, or decrease the viability of, algorithms that depend on
finding a definition of normal and alerting based on deviations from that norm.

A highly sensitive and specific detection of DNS tunnels on a busy network link
is a very important problem in the arena of network security as it enables
administrators to block or otherwise control these potential sources of
compromise. A non-exhaustive, but informative, list of things that DNS tunnels
can be used for is:

\begin{itemize}
\item Data exfiltration
\item Two-way communication
\item Communication through restrictive firewalls
\item Transport layer for complete VPN solutions
\item Arbitrary data transmission
\end{itemize}

In enterprises that deal with sensitive information such as financial, health,
personal or intellectual property information it is of the utmost importance to
control the access to this information. Even if an enterprise does not have
information that needs protecting, DNS tunnels should still be blocked in order
to prevent malware from communicating. Because DNS tunnels can be used for
arbitrary communication, they can be used as command-and-control channels for
botnets or any other malicious system that relies on data transmission. Preventing
botnets from operating is of the best interest for the Internet as a whole and
should be a concern of every user of the Internet.

Existing methods of detection rely on one of two methods, character frequency
and signatures, to detect the presence of a DNS tunnel. The signature-based solutions attempt to identify portions of the tunneling
application data (as opposed to the user data that is sent using the tunnel
application) for which a signature can be built. These signature based
solutions are subject to many of the same problems that plague signature based
anti-virus solutions. For example, if the application changes its communication
schemes, the signatures may no longer be valid and may no longer trigger when
they are intended to. Additionally, signatures cannot be built for applications
that are not available for study, or that aren't known of. For applications
that use a custom communication scheme that has not had a signature built
specifically for it, it is very likely that an existing signature will not be
relevant and will not detect it.

Because signature based schemes are not effective in detecting DNS tunnels in a
zero-day\footnote{\emph{Zero-day} refers to a situation where nothing is
known about the attack as it has never been seen or analyzed before.} situation, another
method is required. Analysis based on character frequency is proposed
in\cite{kenton.born.frequency} by Kenton Born which uses character
frequency analysis under the assumption that normal DNS traffic has a unique
character frequency distribution that can be used as a discriminating fingerprint. Based on analysis of common domain names, a
distribution is obtained that accurately describes normal DNS traffic while
effectively detecting DNS tunnel traffic. Results obtained by Born indicate
that DNS tunnels have an approximately uniform character frequency distribution whereas normal DNS queries do not.

A weakness of the approach proposed by Born is that it relies on the assumption
that DNS tunnel traffic \emph{necessarily} has a character frequency
distribution that is different from that of normal DNS queries. This assumption is not necessarily true, and a proof-of-concept
was developed that demonstrates this fact. The tool performs a
\emph{probabilistic encoding} that takes an arbitrary data source and encodes
it into a stream of characters that conforms to a given distribution. This tool
can be used to modify the output of any DNS tunnel application so that their
output conforms to the distribution that Born found for normal DNS traffic. By
utilizing this transformation, Born's approach fails to detect the DNS traffic
as it becomes, from the viewpoint of his algorithm, indistinguishable from
normal DNS traffic. Details of this are given in \ref{propencode}.

Due to the weaknesses listed above, it is clear that a new approach is
necessary that detects DNS tunnels in a zero-day situation and without the known
weakness involving character frequency and probabilistic encoding. This work
proposes, and demonstrates the effectiveness of, a method of detecting DNS
tunnels that meets these requirements.

The proposed method operates under the assumption that DNS tunnels move more
data than a normal domain. By leveraging this fact, it is possible to detect
any use of DNS to transmit arbitrary data by measuring the amount of data that
is transmitted using a particular DNS domain or subdomain. The details of this
measurement methodology is contained in \ref{datameasurement} and requires
estimating the amount of unique data that is transmitted by analyzing the
queries themselves and not simply counting characters in the query string.

This proposed detection methodology is shown to detect DNS tunnels in as few as
ten packets and continues to be robust in highly hostile detection environments
such as those that contain a great deal of non-tunnel traffic as well as benign
uses of DNS for transmitting arbitrary data\footnote{Many security vendors
utilize the fact that DNS and UDP port 53 are so loosely controlled in order to
ensure that their deployed devices can communicate with the vendor intelligence
unimpeded.}. Detector performance on commodity hardware is shown to scale to
greater than two gigabits of UDP port 53 throughput per second, indicating that
this methodology does not sacrifice performance for detection accuracy and
remains practical for monitoring very large networks.

\newpage
\section{Background}

\subsection{Domain Name System (DNS)}

DNS (Domain Name System) is the service through which names are mapped to
resources. Typically, this maps a name (such as \emph{www.google.ca}) to an IP
address. The value of this service is that names are considerably more
flexible, and considerably easier to remember, than the resource or record that
they point to. For example, \emph{google.com} is considerably
easier to remember than one of the IP addresses that it points to, such as
74.125.226.34. \emph{google.com} is also considerably more flexible, since it
points to not just one but several addresses, and successive responses
will receive different records in a round-robin, or random, fashion. This
rotation of responses allows for a crude form of load balancing and automatic
fail over, while retaining the ease of use.

The DNS protocol is assigned both UDP and TCP port 53 for
communication with most communication operating over UDP as opposed to TCP. The
use of TCP depends on the implementation of the resolver, however the
specification indicates the TCP should be used if the response data exceeds 512
bytes or during a zone transfer\cite{ietf.rfc1035}. DNSSEC (Domain Name System
Security Extensions), due to the fact that it requires a signature of
authenticity for all responses, will often cause the response to
require TCP\cite{iana.rfc4034}. 
% http://www.iana.org/assignments/dns-sec-alg-numbers/dns-sec-alg-numbers.xml
Because there is no \emph{requirement} for when
TCP be used, some resolvers may be implemented to use TCP for all responses as
this does not violate the specifications for DNS.

The experiments related to this work do not consider the situation of DNS over
TCP since the analysis techniques are identical since the formats of the UDP
and TCP responses is identical once the TCP stream is reassembled. Modification
of the tools developed for this work would require the ability to perform TCP
stream reassembly in order to extract the responses from the TCP response.

Because DNS is such an integral component of Internet communication it is not
reasonable to simply block it while still expecting a functional Internet
browsing experience. A common approach is called DNS \emph{proxying} which
forces all DNS queries to be made to a DNS proxy server that is controlled by
the interested entity (ISP, company, etc...). This DNS proxy server is
responsible for handling all DNS queries for the internal network, and any DNS
queries that are destined for the Internet are typically dropped by the
firewall in this type of configuration. The DNS proxy server operates in
\emph{recursive mode}, which means that if a question is asked of it to which
it does not know the answer, the proxy server will then query for the answer (by
issuing its own query to the global DNS system) and then respond to the initial
request using this response.

DNS is a heavily cached protocol due to how often data can be reused between queries. Consider how often a desktop Internet user causes a request for
\emph{google.com}. If a request had to traverse the entire DNS system every
time, this would represent a very considerable amount of traffic being
generated. To avoid this, every DNS record has extra information about it that
includes, among other things, how long that it can be cached for. Standard
caching lengths put it around one hour which means that a DNS server will only
recursively pass on a query for a record once an hour. This caching period is
not constant and can be set differently, depending on the information the
record contains. Some records require a considerably lower TTL (Time To Live),
as low as one minute, while for others a considerably longer (months) may be
appropriate.

This proxy architecture removes some naive operation modes for DNS tunnels
that will be discussed in in more detail in \ref{dnstunnels.types.raw}, but does
not offer any protection against the more common forms of DNS tunnels.

\subsection{Covert Channels}

Covert channels are methods of communication that use non-standard means of
communication for the purpose of evading detection and/or blocking by the
existing security infrastructure. Covert channels may utilize
portions of an existing protocol or communication channel, or they may find
ways of transporting information utilizing a completely new medium. An example
of the latter is called a \emph{timing channel}, which can utilize the timing
between packets to convey information. A timing channel carefully controls the
timing between packets sent to a remote server to encode information, thereby
utilizing a method of communication that is not utilized by any
\emph{legitimate} protocol or communication method.

Covert channels come in may forms and not all types support the properties that
are normally associated with a communication channel. Because they are built on
unorthodox, or unreliable, transmission media and are subject to the effects of
intermediate routing and networking devices they cannot always offer all of the
same functionality as a legitimate channel.

Covert channels need not support bidirectionality due
to either the constraints of the underlying medium, or the effect of
intermediate devices. A covert channel that is only useful for reception is one
that utilizes a third-party image hosting service. It is possible to embed
arbitrary information into the header portion of an otherwise completely benign
JPEG image file which could then be posted to Facebook, Flickr, or any other
publicly accessible image hosting service. This image file can then be checked by the remote
hosts to pull the information however, due to the nature of the image services, the remote
hosts may have no
way of posting information back to the other endpoint, making the
communication channel unidirectional.

Real time data transfer refers to the ability
for a communication channel to send data immediately. UDP, by its very nature,
supports this and TCP supports this via the PUSH flag which indicates that data
is being sent before a full window has been accumulated. The TCP PUSH flag
is used, for example, during an SSH connection in order to provide
interactivity. Timing channels, or any channel that relies on modifying normal
system traffic instead of generating their own traffic, by their nature, are
unable to support real time data transfer. This is because they need to wait for
a system packet in order to send their data, and if the system goes for a period
of time without sending data then the covert channel must wait as well.

\subsection{DNS Tunnels}
\label{dnstunnels.types}

DNS tunneling is the method by which arbitrary data is made to be transferred
over the same channels as DNS. DNS tunnels come in one of two types: raw, or
conforming.

\label{dnstunnels.types.raw}
Raw DNS tunnels do not attempt to mimic or conform to the DNS
specifications, and simply attempt to utilize the fact that UDP port 53 is
often left open in firewalls. Raw tunnels attempt to exploit this by
transmitting arbitrary traffic using UDP port 53 packets with arbitrary payload.
This is the most efficient exploitation of the ubiquity of DNS as it incurs the
lowest amount of overhead, both computationally and in terms of network
throughput. The trade off for this efficiency is that it is the least
conforming and the most likely to get stopped by either a firewall or a proxy.
In the situation where all DNS queries are forced to be proxied through a
dedicated DNS server, raw DNS tunnels will fail to operate as expected. This is
because when the UDP port 53 traffic is redirected to the proxy, the DNS server
will attempt to interpret the arbitrary payload as a DNS packet and will likely
fail. When it fails, it will drop the packet thereby preventing all raw UDP
port 53 communication. Because these types of tunnels are effectively blocked by
standard firewall and proxy practices, detection of these tunnels is not
considered in this work.

\label{dnstunnels.types.conforming}
Conforming DNS tunnels produce DNS packets that conform to all appropriate
specification and RFC documents and, as far as any DNS server, is concerned the
traffic generated is valid DNS traffic. These tunnels incur the highest
computational and throughput overhead, but have the advantage that detecting
and blocking them is a very difficult process. The detection of this type of
DNS tunnels is the topic of this work. This type of tunnel is capable of
operating, even in an environment with very strict firewall and proxy policies.
Because this type of tunnel operates in very hostile (to the operation of the
tunnel) environment, detection of this type of DNS tunnel is of interest to all
levels of government and industry.

Conforming DNS tunnels operate by embedding the data for transmission into the
query string and the response and require a modified, non-conforming, DNS
server on one end of the connection and a piece of software on the client end.
Typically these types of DNS tunnels have one endpoint that is controlled by the
tunnel user, with that controlled endpoint running dedicated server software
that. The client and server software is responsible for transforming arbitrary
information into DNS queries, and for translating DNS responses back into the
arbitrary data that it represents. The precise details of how the translation
is done between DNS and the raw data depends on the implementation.

\subsubsection{DNS Tunnel Software}

Some DNS tunneling software is OzymanDNS\footnote{\url{http://dnstunnel.de/}},
Iodine\footnote{\url{http://code.kryo.se/iodine/}},
Dns2tcp\footnote{\url{http://hsc.fr/ressources/outils/dns2tcp/index.html.en}},
DNScat\footnote{\url{http://tadek.pietraszek.org/projects/DNScat/}} and
DeNiSe\footnote{\url{http://c0re.23.nu/c0de/snap/DeNiSe-current.tar.bz2}}, and
PSUDP\footnote{\url{http://www.kentonborn.com/psudp}}. Each of these have
slightly different operational characteristics, but they all aim to do the same
thing, which is transmission of arbitrary data over DNS.

Iodine supports raw tunnels, as well as moving information back from the server
in A (IPv4 addresses), MX (mail server), TXT (arbitrary text) and may other
supported DNS record types.TXT records are rarely used by
consumers or end-user applications, and so a blanket block policy of TXT
records for end-user devices would have very little impact on end-user
applications. TXT records are as close to a raw tunnel that a conforming
tunnel can get to in terms of throughput.

DNScat utilizes CNAME (alias to
another record) records and a supplementary A record, when appropriate, however
the A record is not actually used for throughput. OzymanDNS and DeNiSe utilize
solely the
TXT record, which is as close to the raw tunnel as possible, however can be
easily neutered by simply blocking TXT records. Because these tools only uses
TXT records it is possible that it is the least flexible and deployable out of
those listed above given a hostile environment. Dns2tcp utilizes with TXT or
KEY records, which makes it as flexible as OzymanDNS and DeNiSe. The KEY DNS
record was designated for specific uses\cite{ietf.rfc2931}, but has been
deprecated now\cite{ietf.rfc3445} in favour of the DNSKEY record for use with
DNSSEC\cite{ietf.rfc3755} and the IPSECKEY for use with
IPSEC\cite{ietf.rfc4025}. Because of this deprecation, use of the KEY record is
subject to strict filtering which greatly reduces the effectiveness of this
solution.

All of the above tools utilize encoding and decoding mechanisms that would
successfully propagate through a proxy server, at the cost of the fact
that the tunnel applications have to generate their own traffic. PSUDP, proposed
by Kenton Born, aims to remove the latter requirement by creating slack space
within an existing DNS packets at the UDP transport layer. He proposes two ways
of creating this slack space: naively placing it at the end of the packet, and
rearranging the DNS query string to utilize pointers to create this
slack space in the middle of the packet. Pointers allow
for the re-use of DNS query strings within a packet to save on space. They are
an optional component of the specification, but allow for considerable space
savings. A pointer in a DNS packet is a special sequence of bytes that
indicates where in the packet the processing of the query string should jump
to. When processing a query string, only a single pointer can be followed,
according to spec, which prevents multiple redirection and infinite loops
(where a pointer points to itself). By having a pointer point forward in the
packet, it is possible to cause the parsing of a query string to skip a number
of bytes, creating slack space.

This method relies on this slack space, which is not parsed by normal servers
or clients, but can be contain arbitrary data that is extracted by special
clients. However, because this slack space is not processed by DNS servers, in
an environment where all DNS queries must go through a proxy, this method is
incapable of producing a covert channel that is able to penetrate a strictly
proxied DNS environment.

\section{Problem Statement}

DNS tunnel detection is a complicated task made worse by the fact that DNS
tunnel traffic can be completely legitimate, conforming network traffic. It
need not violate any established standards or conventions, which makes it
difficult to detect against the background of normal DNS traffic.

This property of DNS tunnels makes them a commonly used transport mechanism
when data exfiltration or network control circumvention is the end goal. For
this reason an efficient method of detecting DNS tunnels is required that can
effectively detect a DNS tunnel against normal DNS traffic with a low
false-positive rate and that must not be susceptible to existing methods of
circumvention.

\section{Review of the State of the Art}

% \section{Problem Statement}
% 
% As networks increase in capacity and more applications are producing traffic
% across the same network connection, the heterogeneity of the traffic on these
% links is increasing. With this increasing diversity in network traffic comes
% an
% ever widening range of acceptable behaviour patterns which results in a very
% hostile and complicated environment for anomaly detection algorithms.
% 
% The goal of this research is to investigate the feasibility of detecting DNS
% tunnels in near real time at high throughput on a complex network link. This
% detection method should also be robust against known methods of avoiding
% existing techniques.

% As computing technology has evolved, the amount of information that needs to be
% protected has grown with it. In 2012, a small ISP (Internet Service Provider)
% that serves less than ten thousand clients can move several terabytes of data
% per day\footnote{YouTube video at a resolution of 720p is encoded at a bitrate
% of approximately 2.5Mbps which translates into approximately fifty megabytes for
% a three minute video. If a average person watches four such videos in a day,
% then ten thousand clients can produce two terabytes of traffic in a day.}.
% 
% Included in this is data generated by hundreds, if not thousands, of unique
% applications including web browsers, operating systems, video games,
% communication programs, office programs and machine-to-machine communications.
% It is important to realize that each web application (GMail, Yahoo mail, Google
% Docs, YouTube, etc...) behaves in this respect as a unique application. Each
% type of traffic from each application has its own specific pattern that
% describes the traffic it produces under normal circumstances and deviations from
% this pattern contain important information.
% 
% The problem of identifying what portions of the data are either malicious or
% otherwise abnormal becomes one of both scale and discrimination. When looking at
% network links sufficiently busy to supply enough information for analysis, there
% is also a great deal of unrelated traffic that needs to be filtered out in order
% to ensure an accurate result. Being discriminatory enough detect relatively rare
% events\footnote{A DNS tunnel, for example, can be as few as ten to twenty
% packets within billions over a day.} without incurring a large number of false
% positive is essential for an effective detection method.
% 
% Covert communication channels can fall into the category of a 'needle in a hay
% stack', or they can be very obvious but this depends on, primarily, how much
% data they are moving. The more data that is being moved across a covert channel,
% the more likely it is to show up as an anomaly due to the number of packets and
% the proportion of the total traffic it represents.
% 
% Covert channels can be formed over any existing protocol or messaging system,
% but need not require one. One example of a covert channel that does not use
% existing protocols or data transmission mechanisms is called a \emph{timing
% channel}. Timing channels utilize some element of timing to encode and transmit
% information. For example, a timing channel can be formed by encoding information
% sent to remote computers by modifying the delay between packets.

% In order to be able to detect these anomalies at a large scale, and to reduce
% the number of devices performing the analysis on the network, the point at which
% the interception is done needs to be moved closer to the centre of the network.
% Throughput on a given link increase rapidly as the chosen link gets closer to
% the centre of the network, due to aggregation of traffic. 
% 
% Sifting through gigabits of traffic becomes a very computationally expensive
% problem; one which is normally solved by applying more, or specialized,
% hardware\footnote{See proprietary solutions, such as those offered by
% TippingPoint, which employ specialized hardware components such as ASICs and
% FPGAs to improve performance. Other vendors, such as Wedge Networks products,
% improve performance by exploiting the ability to perform portions of its
% workload in parallel and add more commodity processors.}. In order to combat
% this inevitable march of requirements, 

\end{document}