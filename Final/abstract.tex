\addcontentsline{toc}{chapter}{Abstract}
\begin{center}
\textbf{\large Abstract}
\end{center}

In a world that runs on data, protection of that data and protection of the \emph{motion} of that data is of the utmost importantance. Covert communication channels attempt to circumvent established methods of control, such as firewalls and proxies, by utilizing non-standard means of getting messages between two endpoints. DNS (Domain Name System), the system that translates text-based resource names into IP addresses, is a very common and effective platform upon which covert channels are built. This work proposes, and demonstrates the effectiveness of, a novel new technique that estimates data transmission throughput over DNS and its weight against the norm for a given network link to identify the existence of a DNS tunnel. The proposed technique is robust in the face of the obfuscation techniques that are able to hide tunnels from existing detection methods.

%This document provides information on how to write your thesis using
%the \LaTeX\ document preparation system.  You can use these files as a
%template for your own thesis, just replace the content, as necessary.
%You should put your real abstract here, of course.

% \vspace{1cm}
% 
% \emph{``The purpose of the abstract, which should not exceed 150 words for
% a Masters' thesis or 350 words for a Doctoral thesis, is to provide
% sufficient information to allow potential readers to decide on relevance
% of the thesis. Abstracts listed in Dissertation Abstracts International
% or Masters' Abstracts International should contain appropriate key
% words and phrases designed to assist electronic searches.''}
% 
% \hfill --- MUN School of Graduate Studies
